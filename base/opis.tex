\documentclass{article} % This command is used to set the type of document you are working on such as an article, book, or presenation

\usepackage{geometry} % This package allows the editing of the page layout
\usepackage{amsmath}  % This package allows the use of a large range of mathematical formula, commands, and symbols
\usepackage{graphicx}  % This package allows the importing of images
\usepackage[polish]{babel}
\usepackage[T1]{fontenc}
\usepackage{graphicx}

\newcommand{\maketitletwo}[2][]{\begin{center}
        \Large{\textbf{SzybkaPaczka}} % Name of course here
        \vspace{5pt}
        
        \normalsize{Jacek Markiewicz, Karol Bielaszka % Your name here
        
        \today}        % Change to due date if preferred
        \vspace{15pt}
        
\end{center}}
\begin{document}
    \maketitletwo[5] 
    
\section*{Tematyka i założone cele}
\noindent Tematem projektu jest baza danych dla firmy przewożącej paczki pomiędzy paczkomatami. Baza przechowuje informacje dotyczące klientów, pracowników, pojazdów oraz przesyłanych paczek. Aplikacja będzie udostępniać interfejs umożliwijący komunikację pomiędzy bazą a klientami, kurierami i aplikacją do zarządzania logistyką (aplikacji logistycznej nie będziemy implementować). 
\section*{Schemat bazy}
    Schemat bazy możemy podzielić na trzy główne części:
    \begin{enumerate}
        \item \textbf{Osoby} - składa się z osób, uprawnień jakie posiadają, stanowisk jakich pracują oraz informacji o stawieniu się pracownika w pracy i zakończeniu pracy w danym dniu.
        \item \textbf{Pojazdy} - składa się z pojazdów, które posiada firma, ich rodzajów, pojemności, wymaganych serwisów (uzależnionych od rodzaju pojazdu), przeprowadzonych serwisów, usterek i ich napraw oraz wymaganych uprawnień do prowadzenia pojazdu.
        \item \textbf{Zlecenia i kursy} - składają się z informacji o zleceniach i transporcie paczek pomiędzy paczkomatami. Dokładny opis przewozu paczki znajduje się poniżej.
    \end{enumerate}
\noindent 
\section*{Historia zlecenia}
    \begin{enumerate}
        \item Zgłoszenie zlecenia przez klienta, odnotowane w tabeli zlecenia;
        \item Włożenie paczki przez klienta do skrytki, odnotowane w nadaniach;
        \item Zabranie paczki przez kuriera (odnotowane w zlecenia\_kursy) i przewiezienie do magazynu (odnotowane w zlecenia\_magazyny) do którego przynależy dany paczkomat;
        \item Następnie kilkukrotne przewiezenie paczki pomiędzy różnymi magazynami (odnotowane w zlecenia\_kursy i zlecenia\_magazyny);
        \item W międzyczasie aplikacja logistyczna rezerwuje pustą skrytkę w paczkomacie, do którego ma trafić przesyłka (odnotowane w odbiory, obie daty równe NULL, jest to równoznaczne z zarezerwowaniem skrytki na dane zlecenie);
        \item Zawiezienie paczki do odpowiedniego paczkomatu i włożenie paczki do zarezerwowanej skrytki (odnotowane w odbiory jako data\_dostarczenia);
        \item Odebranie paczki z paczkomatu przez odbiorcę (odnotowane w odbiory jako data\_odbioru).
    \end{enumerate}
\section*{Historia kursu}
    \begin{enumerate}
        \item Warstwa logistyczna tworzy zlecenie do którego przyporządkowuje magazyn początkowy (magazyn\_start), magazyn docelowy, kierowcę i pojazd znajdujących się w magazynie\_start, każdy kurs odbywa się pomiędzy dwoma magazynami (być może między jednym i tym samym);
        \item Do pojazdu przypisanego do kursu zostają zapakowane odpowiednie paczki (odnotowane w zlecenia\_kursy);
        \item Kurs wyjeżdża z magazynu (odnotowane w kursy jako data\_wyjazdu);
        \item Jeżeli kurs zbiera/rozwozi paczki z/do paczkomatów:
            \begin{itemize}
                \item Kurs zatrzymuje się przy paczkomacie (odnotowane w kursy\_paczkomaty jako data\_przyjazdu);
                \item Następuje przepakowanie odpowiednich paczek (odnotowane w zlecenia\_kursy lub odbiory);
                \item Kurs odjeżdża od paczkomatu i rusza w dalszą trasę (odnotowane w kursy\_paczkomaty jako data\_odjazdu);
            \end{itemize}
        \item Kurs przyjeżdża do docelowego magazynu (odnotowane w kurs jako data\_przyjazdu);
        \item Następuje wypakowanie wszystkich paczek do magazynu (odnotowane w zlecenia\_magazyny);
        \item Kierowca deklaruje, że cały pojazd został rozpakowany (odnotowane w deklaracja\_wypakowan), nie możemy tego wnioskować z tabeli zlecenia\_magazyny, ponieważ paczka mogła się zgubić podczas transportu.
    \end{enumerate}
\section*{Problemy i postawione założenia}
	\begin{itemize}
	\item Zdecydowaliśmy nie zajmować się systemem wypłat, oraz harmonogramem pracowników.
	\item Początkowo chcieliśmy trzymać graf magazynów i obliczać optymalną trasę dla każdego zlecenia, jednak zdecydowaliśmy pozostawić to warstwie logistycznej, która mogła by pracować na naszej bazie.				
    \item Zakładamy, że każdy magazyn może pomieścić nieskończoną liczbę paczek.
    \item Zakładamy, że pesel jest niezmienialny.
	\item Zakładamy, że rejestracja pojazdu jest niezmienialna.
	\item Jeśli pojazd z pewnych przyczyn zniknie (np. zostanie ukradziony w trakcie odbywania kursu) to jest to traktowane jako usterka (której naprawą jest odzyskanie pojazdu). 
	\end{itemize}
\section*{Plany}
    Wprowadzimy indeksy do wyszukiwania id\_zlecenia, w większości tabel je zawierających, być może również id\_osoby lub (id\_osoby,id\_magazynu) do praca\_osoby i id\_pojazdu do serwis. Rozpatrzymy również dodanie perspektyw w celu ułatwienia dostępu do danych w bazie. Rozważymy wprowadzenie zmienialnej rejestracji pojazdów.
\end{document}